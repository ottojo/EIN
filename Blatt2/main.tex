\documentclass{article}

\usepackage[utf8]{inputenc}
\usepackage{enumitem}
\usepackage{float}
\usepackage{circuitikz}
\usepackage{todonotes}
\usepackage{amsmath}
\usepackage{tikz}
\usetikzlibrary{shapes, calc, shapes, arrows}

\title{Blatt 2}
\author{Luca Krüger, Jonas Otto, Jonas Merkle (Gruppe R)}

\begin{document}
\maketitle

\section{Boole'sche Funktionen}
\subsection{}
\label{gates}
\begin{enumerate}[label=\alph*)]
  \item AND-Gatter: \\
        $w_1=w_2=w_3=1, \theta=3$

  \item OR-Gatter: \\
        $w_1=w_2=w_3=1, \theta=1$
\end{enumerate}


\subsection{}
Nein die Darstellung ist nicht eindeutig. Gewichtsvektor und Schwellenwert könnten auch z.B: so gewählt werden:
\begin{enumerate}[label=\alph*)]
  \item AND-Gatter:\\
        $w_1, w_2, w_3$ beliebig, $\theta = \sum\limits_k w_k$

  \item OR-Gatter:\\
        $w_1=w_2=w_3=w_4, \theta = \min{w_k}$
\end{enumerate}


\subsection{}
\begin{enumerate}[label=\alph*)]
  \item Boolsche Funktion
        \begin{figure}[H]
          \centering
          \tikzstyle{inputNode}=[draw,circle,minimum size=10pt,inner sep=0pt]
          \tikzstyle{stateTransition}=[->, thick]
          \begin{tikzpicture}

            \node[draw,circle,minimum size=40pt,inner sep=0pt] (s1) at (0,-1) {$\Sigma \geq 1$};
            \node[draw,circle,minimum size=40pt,inner sep=0pt] (s2) at (3, 0.5) {$\Sigma \geq 2$};

            \node[inputNode] (x1) at (-3, 2) {$\tiny x_1$};
            \node[inputNode] (x2) at (-3, 0) {$\tiny x_2$};
            \node[inputNode] (x3) at (-3, -2) {$\tiny x_3$};

            \draw[stateTransition] (x1) to[out=0,in=180] node [midway,above] {$1$} (s2);
            \draw[stateTransition] (x2) to[out=0,in=150] node [midway,above] {$1$} (s1);
            \draw[stateTransition] (x3) to[out=0,in=210] node [midway,above] {$1$} (s1);
            \draw[stateTransition] (s1) to[out=0,in=180] node [midway,below=3pt] {$1$} (s2);


            \draw[stateTransition] (s2) -- ++(2,0) node [right=1pt] {$V(\boldsymbol{x})$};
          \end{tikzpicture}
          \caption{Aufgabe 2, 1.}
        \end{figure}
  \item
        Mit Gewichten $\boldsymbol{w}=(2\ 1\ 1)$ und Schwellwert $\theta=3$ genügt ein Schwellwertneuron.
\end{enumerate}

\section{Schwellwertneuronen}
\subsection{}
\begin{figure}[H]
  \centering
  \tikzstyle{inputNode}=[draw,circle,minimum size=10pt,inner sep=0pt]
  \tikzstyle{stateTransition}=[->, thick]
  \begin{tikzpicture}

    \node[draw,circle,minimum size=40pt,inner sep=0pt] (s1) at (0,2) {$\Sigma \geq 0.5$};
    \node[draw,circle,minimum size=40pt,inner sep=0pt] (s2) at (0,0) {$\Sigma \geq 0.25$};
    \node[draw,circle,minimum size=40pt,inner sep=0pt] (s3) at (0,-2) {$\Sigma \geq 0.25$};

    \node[inputNode] (x1) at (-2, 2) {$\tiny x_1$};
    \node[inputNode] (x2) at (-2, 0) {$\tiny x_2$};
    \node[inputNode] (x3) at (-2, -2) {$\tiny x_3$};

    \draw[stateTransition] (x1) to[out=0,in=180] node [midway, sloped, above] {} (s1);
    \draw[stateTransition] (x2) to[out=0,in=180] node [midway, sloped, above] {} (s2);
    \draw[stateTransition] (x3) to[out=0,in=180] node [midway, sloped, above] {} (s3);

    \node[draw,circle,minimum size=40pt,inner sep=0pt] (s4) at (3,0) {$\Sigma \geq 3$};

    \draw[stateTransition] (s1) to[out=0,in=150] node [pos=.7,above=5pt] {$2$} (s4);
    \draw[stateTransition] (s2) to[out=0,in=180] node [pos=.7,above] {$1$} (s4);
    \draw[stateTransition] (s3) to[out=0,in=210] node [pos=.7, below=5pt] {$1$} (s4);

    \draw[stateTransition] (s4) -- (5,0) node [right=1pt] {$y$};
  \end{tikzpicture}
  \caption{Aufgabe 2, 1.}
\end{figure}

\subsection{}
\begin{figure}[H]
  \centering
  \tikzstyle{inputNode}=[draw,circle,minimum size=10pt,inner sep=0pt]
  \tikzstyle{stateTransition}=[->, thick]
  \begin{tikzpicture}

    \node[draw,circle,minimum size=40pt,inner sep=0pt] (s1) at (0,1) {$\Sigma \geq 1$};
    \node[draw,circle,minimum size=40pt,inner sep=0pt] (s2) at (0,-1) {$\Sigma \geq -1$};
    \node[draw,circle,minimum size=40pt,inner sep=0pt] (s3) at (3,0) {$\Sigma \geq 2$};

    \node[inputNode] (x1) at (-3, 2) {$\tiny x_1$};
    \node[inputNode] (x2) at (-3, 0) {$\tiny x_2$};
    \node[inputNode] (x3) at (-3, -2) {$\tiny x_3$};

    \draw[stateTransition] (x1) to[out=0,in=150] node [midway, sloped, above] {} (s1);
    \draw[stateTransition] (x2) to[out=0,in=180] node [midway, sloped, above] {} (s1);
    \draw[stateTransition] (x3) to[out=0,in=210] node [midway, sloped, above] {} (s1);

    \draw[stateTransition] (x1) to[out=0,in=150] node [pos=.99, above=2pt] {$-1$} (s2);
    \draw[stateTransition] (x2) to[out=0,in=180] node [pos=.9, above=1pt] {$-1$} (s2);
    \draw[stateTransition] (x3) to[out=0,in=210] node [pos=.99, below=2pt] {$-1$} (s2);

    \draw[stateTransition] (s1) to[out=0,in=150] node [pos=.7,above=5pt] {$1$} (s3);
    \draw[stateTransition] (s2) to[out=0,in=210] node [pos=.7,below=5pt] {$1$} (s3);

    \draw[stateTransition] (s3) -- (5,0) node [right=1pt] {$V(\boldsymbol{x})$};
  \end{tikzpicture}
  \caption{Aufgabe 2, 2.}
\end{figure}

\section{Logistisches Neuron}
\subsection{}
\begin{enumerate}[label=\alph*)]
  \item Eine Erhöhung von $w_1$ bewirkt ein schnelleres Ansteigen des Ausgangs gegen 1.
  \item Bei $w_1 < 0$ strebt der Ausgang gegen 0, statt gegen 1.

  \item
        \begin{tabular}{rcc}
          \hline
                          & $w_1 > 0$     & $w_1 < 0$     \\
          \hline
          $b_1\uparrow$   & $\leftarrow$  & $\rightarrow$ \\
          $b_1\downarrow$ & $\rightarrow$ & $\leftarrow$  \\
          \hline
        \end{tabular}
\end{enumerate}

\subsection{}
% w1 = -1
% b1 = -1
% w2 = 1
% TODO: check if equivalent
$w_1 = 1, w_2 = -1$\\
$b_1=b_2 = 1$

\end{document}
